% Options for packages loaded elsewhere
\PassOptionsToPackage{unicode}{hyperref}
\PassOptionsToPackage{hyphens}{url}
%
\documentclass[
]{book}
\usepackage{amsmath,amssymb}
\usepackage{lmodern}
\usepackage{iftex}
\ifPDFTeX
  \usepackage[T1]{fontenc}
  \usepackage[utf8]{inputenc}
  \usepackage{textcomp} % provide euro and other symbols
\else % if luatex or xetex
  \usepackage{unicode-math}
  \defaultfontfeatures{Scale=MatchLowercase}
  \defaultfontfeatures[\rmfamily]{Ligatures=TeX,Scale=1}
\fi
% Use upquote if available, for straight quotes in verbatim environments
\IfFileExists{upquote.sty}{\usepackage{upquote}}{}
\IfFileExists{microtype.sty}{% use microtype if available
  \usepackage[]{microtype}
  \UseMicrotypeSet[protrusion]{basicmath} % disable protrusion for tt fonts
}{}
\makeatletter
\@ifundefined{KOMAClassName}{% if non-KOMA class
  \IfFileExists{parskip.sty}{%
    \usepackage{parskip}
  }{% else
    \setlength{\parindent}{0pt}
    \setlength{\parskip}{6pt plus 2pt minus 1pt}}
}{% if KOMA class
  \KOMAoptions{parskip=half}}
\makeatother
\usepackage{xcolor}
\IfFileExists{xurl.sty}{\usepackage{xurl}}{} % add URL line breaks if available
\IfFileExists{bookmark.sty}{\usepackage{bookmark}}{\usepackage{hyperref}}
\hypersetup{
  pdftitle={Tools for human-screen interactions},
  pdfauthor={cjlortie},
  hidelinks,
  pdfcreator={LaTeX via pandoc}}
\urlstyle{same} % disable monospaced font for URLs
\usepackage{longtable,booktabs,array}
\usepackage{calc} % for calculating minipage widths
% Correct order of tables after \paragraph or \subparagraph
\usepackage{etoolbox}
\makeatletter
\patchcmd\longtable{\par}{\if@noskipsec\mbox{}\fi\par}{}{}
\makeatother
% Allow footnotes in longtable head/foot
\IfFileExists{footnotehyper.sty}{\usepackage{footnotehyper}}{\usepackage{footnote}}
\makesavenoteenv{longtable}
\usepackage{graphicx}
\makeatletter
\def\maxwidth{\ifdim\Gin@nat@width>\linewidth\linewidth\else\Gin@nat@width\fi}
\def\maxheight{\ifdim\Gin@nat@height>\textheight\textheight\else\Gin@nat@height\fi}
\makeatother
% Scale images if necessary, so that they will not overflow the page
% margins by default, and it is still possible to overwrite the defaults
% using explicit options in \includegraphics[width, height, ...]{}
\setkeys{Gin}{width=\maxwidth,height=\maxheight,keepaspectratio}
% Set default figure placement to htbp
\makeatletter
\def\fps@figure{htbp}
\makeatother
\setlength{\emergencystretch}{3em} % prevent overfull lines
\providecommand{\tightlist}{%
  \setlength{\itemsep}{0pt}\setlength{\parskip}{0pt}}
\setcounter{secnumdepth}{5}
\usepackage{booktabs}
\ifLuaTeX
  \usepackage{selnolig}  % disable illegal ligatures
\fi
\usepackage[]{natbib}
\bibliographystyle{apalike}

\title{Tools for human-screen interactions}
\author{cjlortie}
\date{}

\begin{document}
\maketitle

{
\setcounter{tocdepth}{1}
\tableofcontents
}
\hypertarget{screen-adaptation-theory}{%
\chapter{Screen adaptation theory}\label{screen-adaptation-theory}}

\includegraphics[width=4in,height=\textheight]{./sat.png}

\hypertarget{context}{%
\subsection*{Context}\label{context}}
\addcontentsline{toc}{subsection}{Context}

Screens are a portal to information and to one another. \href{https://datareportal.com/global-digital-overview}{Nearly 5 billion people as of 2022 use the internet}. Treating screens and digital time only as a pathology neglects the inherent capacity for screens as a tool to promote higher levels of performance and novel approaches to problem solving. \href{https://www.frontiersin.org/articles/10.3389/fpsyg.2017.01335/full}{Mental models and the associated cognitive architecture} that we frame conceptually to decision making is critical for better choices. The screen adaptation theory (SAT) is proposed herein as a heuristic to enable individuals to use evidence and structured thinking in approaching screen time decisions.

Screens are a place. Going to work, visiting friends, visiting the library, and many other key personal tasks and professional functions are done via screens. It can seem trivial, but labeling these choices, explicitly, provides a sense of coherence and purpose. Screens as a place also provides context and ecology. Adaptation is the sum of traits that an individual possesses or develops that promote survival or higher levels of relative performance without those traits. The goal must be to amplify and identify traits that mitigate the real costs of screens and tip the net sum to positive and higher levels of performance. Theory is a set of principles. Many discipline support screens as a place and an adaptationist programme for screen time use. Source theory and scientific evidence depending the on specific context and choice.

\hypertarget{learning-outcomes}{%
\subsection*{Learning outcomes}\label{learning-outcomes}}
\addcontentsline{toc}{subsection}{Learning outcomes}

\begin{enumerate}
\def\labelenumi{\arabic{enumi}.}
\tightlist
\item
  Develop a new mental model for screen time.\\
\item
  Explore and track decisions associated with screen time.\\
\item
  Examine individual costs and benefits of screens.
\end{enumerate}

\hypertarget{schedule}{%
\subsection*{Schedule}\label{schedule}}
\addcontentsline{toc}{subsection}{Schedule}

Here is an outline of the challenges proposed to explore these principles in this course.

\begin{tabular}{rll}
\toprule
challenge & focus & tasks\\
\midrule
1 & Screen adaptation theory & read screen adaptation theory, explore a few theories that support change in screen use, and track screen time for a week\\
2 & Ten simple rules & test interventions, individually for your personal and professional interactions with screens for a week\\
3 & How to do nothing & explore liminal time and spaces to recharge and reframe how you work and interact, test out an ecological paradigm for yourself\\
\bottomrule
\end{tabular}

\hypertarget{citation}{%
\subsection*{Citation}\label{citation}}
\addcontentsline{toc}{subsection}{Citation}

Lortie, CJ (2021): Nature hacks for life. figshare. Book. \url{https://doi.org/10.6084/m9.figshare.16879312.v1}

\hypertarget{license}{%
\subsection*{License}\label{license}}
\addcontentsline{toc}{subsection}{License}

This work is licensed under a Creative Commons Attribution-NonCommercial-ShareAlike 4.0 International License.

\hypertarget{challenge-time}{%
\subsection*{Challenge time}\label{challenge-time}}
\addcontentsline{toc}{subsection}{Challenge time}

Use \href{https://scholar.google.com}{Google Scholar} and do a few search with screen time and \ldots. for whatever personal challenge is most urgent. Screen time and memory or fatigue or focus or vision etc. Use the filter tool on the left to return hits from 2018 onwards.

Read screen adaptation theory at Ideas in Ecology and Evolution.

Track screen time use even cursorily. There are digital tools and functions included in the operating system of many devices. Or, go old school and have fun with it using a kitchen timer, stopwatch, or clock.

\hypertarget{reflection-questions}{%
\subsection*{Reflection questions}\label{reflection-questions}}
\addcontentsline{toc}{subsection}{Reflection questions}

\begin{enumerate}
\def\labelenumi{\arabic{enumi}.}
\tightlist
\item
  Did any of the work associated with screen time resonate with your challenge? If so, did the evidence nudge or shift your model and thinking?
\item
  Do you use mental models for other dimensions of your life such as training, sleep, or performance? Does your employer or team adopt models for performance?
\item
  Did tracking confirm your assumptions on frequency and duration of screen time?
\end{enumerate}

\hypertarget{rules}{%
\chapter{Ten simple rules}\label{rules}}

\includegraphics[width=3in,height=\textheight]{./rules.png}

\hypertarget{context-1}{%
\subsection*{Context}\label{context-1}}
\addcontentsline{toc}{subsection}{Context}

Rules help.

\hypertarget{learning-outcomes-1}{%
\subsection*{Learning outcomes}\label{learning-outcomes-1}}
\addcontentsline{toc}{subsection}{Learning outcomes}

\begin{enumerate}
\def\labelenumi{\arabic{enumi}.}
\tightlist
\item
  Explore a checklist of tools or hacks from nature for performance.\\
\item
  Challenge your own absurdity and drives.\\
\item
  Develop a nature identity that includes active engagement with an outdoor pursuit or place.
\end{enumerate}

\hypertarget{challenge-time-1}{%
\subsection*{Challenge time}\label{challenge-time-1}}
\addcontentsline{toc}{subsection}{Challenge time}

\begin{enumerate}
\def\labelenumi{\arabic{enumi}.}
\tightlist
\item
  Review this \href{https://figshare.com/articles/presentation/Nature_hacks_for_life/16878808}{slide deck} and attend discussion.\\
\item
  Read \href{https://www.goodreads.com/book/show/157993.The_Little_Prince}{`The Little Prince'} short tale.\\
\item
  Read \href{https://ojs.library.queensu.ca/index.php/IEE/article/view/15122}{`The Little Prince is an ecologist'} comment paper.\\
\item
  \href{http://www.testmycreativity.com}{Test your creativity} using this short test.\\
\item
  Reflect on your scores and the radial plot and how connectedness can enhance some of the measures.
\end{enumerate}

\hypertarget{reflection-questions-1}{%
\subsection*{Reflection questions}\label{reflection-questions-1}}
\addcontentsline{toc}{subsection}{Reflection questions}

\begin{enumerate}
\def\labelenumi{\arabic{enumi}.}
\tightlist
\item
  Interactions are fundamental to all living organisms. To what extent does interaction theory, very broadly speaking, inform the ecology of your life?\\
\item
  Do you have an outdoor identity? If you could change this view, what would you innovate or augment for this self-vision? Even one mountain climbed makes you a climber. Or one bird spotted and identified a small step to becoming a birder.\\
\item
  Are there some of the nature hacks proposed (or new alternatives you envision) that can be used to restore, recharge, or rev up your creative performance and cognitive clarity?
\end{enumerate}

\hypertarget{nothing}{%
\chapter{Do nothing}\label{nothing}}

\includegraphics[width=3in,height=\textheight]{./nothing.png}

\hypertarget{context-2}{%
\subsection*{Context}\label{context-2}}
\addcontentsline{toc}{subsection}{Context}

Doing nothing is an active process. It is not easy to `do' at all.

\hypertarget{learning-outcomes-2}{%
\subsection*{Learning outcomes}\label{learning-outcomes-2}}
\addcontentsline{toc}{subsection}{Learning outcomes}

\begin{enumerate}
\def\labelenumi{\arabic{enumi}.}
\tightlist
\item
  Explore some of the hacks \href{https://figshare.com/articles/presentation/Nature_hacks_for_life/16878808}{suggested previously} in this course of study.\\
\item
  Contrast at least three options for boosted performance from nature ninja thinking.\\
\item
  Develop a nature identity.
\end{enumerate}

\hypertarget{challenge-time-2}{%
\subsection*{Challenge time}\label{challenge-time-2}}
\addcontentsline{toc}{subsection}{Challenge time}

\begin{enumerate}
\def\labelenumi{\arabic{enumi}.}
\tightlist
\item
  Practice passive `how low can you go' outdoor time. Try for only 12mins total per day outside. Sit outside to recover, and apply a diffuse focus to natural observation.\\
\item
  Download \href{https://www.inaturalist.org}{iNaturalist app}. For one week, identify a single plant, bird, or other animal daily. Alternatively, download any \href{https://www.hortibiz.com/newsitem/news/9-best-plant-identification-app-choices-of-2020/}{nature ID app} such as \href{https://www.picturethisai.com}{PictureThis}. Do the identifications and just keep track yourself in a notebook.\\
\item
  Explore green exercise. Take a meeting outside and walk, take a call outside, walk somewhere new a few times a week, do a short run, jog for 12 mins only, or stretch outside.
\item
  Read the paper \href{https://www.fs.usda.gov/treesearch/pubs/38787}{`Understanding the transformative aspects of the Wilderness and Protected Lands experience upon human health'}. Reflect on small or large changes in how you select where you go or where you engage with natural space. It does not have to be a park or pristine protected area. It has be special only to you. There is excellent evidence that any space that you call your `own' and feel connectedness to enhances health.\\
\item
  Retake this \href{http://www.testmycreativity.com}{creativity test} or try the \href{https://www.ideo.com/blog/build-your-creative-confidence-thirty-circles-exercise}{thirty-circles test}, outside with a clipboard, and show the outcome to someone else for fun feedback.
\end{enumerate}

\hypertarget{reflection-questions-2}{%
\subsection*{Reflection questions}\label{reflection-questions-2}}
\addcontentsline{toc}{subsection}{Reflection questions}

\begin{enumerate}
\def\labelenumi{\arabic{enumi}.}
\tightlist
\item
  Was there a category of outdoor interaction that most suited your needs and was more easily reconciled with existing routines? Passive recovery time, active natural observation, or green exercise (instead of some of the time you might allocate to the gym or other workout times).\\
\item
  Did an identity such as birder, walker, open-air thinker, outdoor meeting person, plant lover, or outdoor reflection seem like a good fit?\\
\item
  What impediments or frictions long-term will present challenges to nature ninja hacks - at the fine-grain resolution of daily experiential boosts?\\
\item
  Given the evidence presented previously for reciprocal restoration benefits and feedback loops between people and natural systems when we `fix up' nature, how can you consider taking outdoor experiential interactions to this next level? This would not be at at the daily level but instead include monthly park or beach cleanups, native planting, weed removal, tending your plants indoors or outdoors weekly, or other processes that demonstrate and affirm benefits to both systems - you and the other natural systems.
\end{enumerate}

  \bibliography{book.bib,packages.bib}

\end{document}
